%%%%%%%%%%%%%%%%%%%%%%%%%%%%%%%%%%%%%%%%%
% Barbara Liskov Presentation
% LaTeX Template
% 23/11/2020
% 
%%%%%%%%%%%%%%%%%%%%%%%%%%%%%%%%%%%%%%%%%

%------------------------------------------------------------------
%	PACKAGES AND THEMES
%------------------------------------------------------------------

\documentclass{beamer}

\mode<presentation> {

\usetheme{Madrid}

}
\hypersetup{
    colorlinks=true,
    linkcolor=blue,
    filecolor=magenta,      
    urlcolor=cyan,
}
\usepackage{graphicx} 
\usepackage{booktabs} 
\usepackage{hyperref} 
\usepackage{listings}

\usepackage{xcolor}

\definecolor{codegreen}{rgb}{0,0.6,0}
\definecolor{codegray}{rgb}{0.5,0.5,0.5}
\definecolor{codepurple}{rgb}{0.58,0,0.82}
\definecolor{backcolour}{rgb}{0.95,0.95,0.92}



\lstdefinestyle{mystyle}{
    backgroundcolor=\color{backcolour},   
    commentstyle=\color{codegreen},
    keywordstyle=\color{magenta},
    numberstyle=\tiny\color{codegray},
    stringstyle=\color{codepurple},
    basicstyle=\ttfamily\footnotesize,
    breakatwhitespace=false,         
    breaklines=true,                 
    captionpos=b,                    
    keepspaces=true,                 
    numbers=left,                    
    numbersep=5pt,                  
    showspaces=false,                
    showstringspaces=false,
    showtabs=false,                  
    tabsize=2
}

\lstset{style=mystyle}

%-----------------------------------------------------------------
%	TITLE PAGE
%-----------------------------------------------------------------

\title[Research Methods]{Barbara Liskov}
\author{Group N}
\institute[GMIT]
{
\textit{Grace Keane} \\\textit{Shirin Nagle} \\ 
\medskip
}
\date{}

\begin{document}

\begin{frame}
\titlepage % Print the title page as the first slide
\end{frame}

%------------------------------------------------------------------
%	BRIEF DESCRIPTION
%------------------------------------------------------------------

\begin{frame}
\frametitle{Brief Description} 

\includegraphics[scale=0.4]{Barbara Liskov}
\centering

\end{frame}


%----------------------------------------------------------------
%	PRESENTATION SLIDES
%----------------------------------------------------------------

\begin{frame}
\frametitle{Important Details}
\begin{itemize}
\item Computer Scientist
\item Born: November 7th, 1939, California
\item One of the first women in the USA to be awarded a PhD in Computer Science
\item She is one of the world's leading authorities on computer language and system design
\item She won numerous awards as well as the Turing award in 2009
\item Since 1966, 70 computer scientists have won the Turing Award. Only 3 have been women. Therefore this is an amazing achievement
\item Created the Liskov's substitution principle
\end{itemize}
\end{frame}

%------------------------------------------------

%------------------------------------------------
\begin{frame}
\frametitle{Turing Award}
Barbara Liskov was awarded the Alan Turing award for  contributions "of lasting and major technical importance to the computer field". 

\vspace{5mm} %5mm vertical space

Awarded for contributions to practical and theoretical foundations of programming language and system design, especially related to data abstraction, fault tolerance, and distributed computing. 

\vspace{5mm}

\end{frame}

%------------------------------------------------

\begin{frame}
\frametitle{Contributions}
\begin{itemize}
\item Venus \& variable scope
\item CLU Programming Language
\item Data Abstraction
\item Liskov Substitution
\end{itemize}
\end{frame}
%------------------------------------------------
\begin{frame}
\frametitle{Venus \& Variable Scope}
Created the “Venus Computer” tailored to supporting the construction of complex software. The Venus operating system was a small time sharing system for the Venus machine used to experiment with how different architectures helped or hindered this process.

\vspace{5mm}

The Venus Operating System is an experimental multiprogramming system which supports five or six concurrent users on a small computer. The system is defined by a combination of microprograms and software.

\vspace{5mm}

Variable scope
\end{frame}
%------------------------------------------------
%------------------------------------------------
\begin{frame}
\frametitle{Data Abstraction}
What we desire from an abstraction is a mechanism which permits the expression of relevant details and the suppression of irrelevant details. In the case of programming, the use which may be made of an abstraction is relevant; the way in which the abstraction is implemented is irrelevant. — Programming With Abstract Data Types - B.Liskov 1974

This work resulted in CLU a programming language which is a predecessor to many OOP languages.
\vspace{5mm}

Her contributions have influenced advanced system developments and set a standard for clarity and usefulness

\end{frame}
%------------------------------------------------

%------------------------------------------------
\begin{frame}
\frametitle{CLU Programming Language}

At MIT she led the design and implementation of the CLU programming language, which emphasized the notions of modular programming, data abstraction, and polymorphism. These concepts are a foundation of object-oriented programming used in modern computer languages such as Java and C\#.

\vspace{5mm}

CLU introduced many features that are used widely now, and is seen as a step in the development of object-oriented programming 

\vspace{5mm}

also notable for its use of classes with constructors and methods, but without inheritance.

\end{frame}
%------------------------------------------------

\begin{frame}
\frametitle{Barbara Liskov's definition}

Developed a new notion of sub typing now known as the Liskov Substitution principle.

\vspace{5mm} %5mm vertical space

At a high level, the LSP states that in an object-oriented program, if we substitute a superclass object reference with an object of any of its subclasses, the program should not break.\\

\vspace{5 mm}

"If for each object o1 of type S there is an object o2 of type T such that for all programs P defined in terms of T, the behavior of P is unchanged when o1 is substituted for o2 then S is a subtype of T"


\end{frame}



%------------------------------------------------

\vspace{5mm}

% Putting code on screen & formatted nicely
\begin{lstlisting}[language=Java]

// Animal super class 
public static class Animal {
  public String favoriteFood;
  public Animal(String favoriteFood) {
    this.favoriteFood = favoriteFood;
  }
}
// Subclass
public static class Dog extends Animal {
  public Dog(String favoriteFood) {
    super(favoriteFood);
  }
}

// Subclass
public static class Cat extends Animal {
  public Cat(String favoriteFood) {
    super(favoriteFood);
  }
}
\end{lstlisting}

\vspace{5mm}

%------------------------------------------------

%------------------------------------------------

\vspace{5mm}

% Putting code on screen & formatted nicely
\begin{lstlisting}[language=Java]
// Method to give treats
public static void GiveTreatTo(Animal animal) {
  String msg = "You fed the " + animal.getClass().getSimpleName() + " some "  + animal.favoriteFood;
  System.out.println(msg);
}

// Assigning treats to animals
// Do not have to create a new method per animal because of the LSP principle
public static void main(String[] args) {
  Dog rover = new Dog("bacon");
  Cat bingo = new Cat("fish");

  GiveTreatTo(rover);
  GiveTreatTo(bingo);
}


Command prompt output:

You gave the Dog some bacon
You gave the Cat some fish
\end{lstlisting}

%------------------------------------------------

\begin{frame}
\frametitle{SOLID} 

\includegraphics[scale=0.8]{SOLID}
\centering


\end{frame}

%------------------------------------------------

\begin{frame}
\frametitle{Principle naming}
She did not name the Liskov Substitution Principle. Apparently, she received an email in the 90’s by somebody asking her whether he got her principle right, surprising her. She had not known that the principle had borne her name for years in the community.
\end{frame}

%------------------------------------------------
\begin{frame}{Conclusion}
\begin{itemize}
\item Subsequent work has mainly been in the area of distributed systems. 
\item Her research has covered many aspects of OS and computation \begin{itemize}
    \item work on object-oriented database systems
    \item  garbage collection
    \item caching, persistence, recovery, 
    \item security, decentralized information flow
    \item modular upgrading of distributed systems, geographic routing
    \item fault tolerance and practical Byzantine fault tolerance
\end{itemize}  
\item Many of these, deal with situations where a complex system fails in arbitrary ways. 
\item Liskov developed methods to allow correct operation even when some components are unreliable.
\item At 81, Barbara Liskov is still active today contributing to and writing many papers as recently as this year.  

\end{itemize}
\end{frame}



%------------------------------------------------

\begin{frame}
\Huge{\centerline{The End}}
\end{frame}

%----------------------------------------------------------------------------------------

\begin{frame}
\frametitle{Clickable References}
\begin{itemize}
\item \href{https://amturing.acm.org/award_winners/liskov_1108679.cfm}{ A.M. Turing; Barbara Liskov;} 

\item \href{https://reflectoring.io/lsp-explained/}{ Reflectoring; The Liskov Substitution Principle Explained;}

\item \href{https://www.geeksforgeeks.org/solid-principle-in-programming-understand-with-real-life-e}{GeeksforGeeks; SOLID Principle in Programming: Understand With Real Life Examples;}


\item \href{https://dev.to/erikwhiting88/liskov-substitution-principle-in-3-minutes-2}{DEV; Liskov Substitution Principle in 3 Minutes;}
\item \href {https://dl.acm.org/doi/abs/10.1145/155360.155367}{A history of CLU Barbara Liskov (1992)}
\item \href{https://dblp.uni-trier.de/pid/l/BarbaraLiskov.html}{Barbara Liskov Index of publications}

\item \href{http://www.pmg.csail.mit.edu/~liskov/newcv-09.pdf}{Barbara Liskov CV}

\item \href{http://www.pmg.csail.mit.edu/~liskov/}{Barbara Liskov home page}

\end{itemize}
\end{frame}

%-----------------------------------------------

\begin{frame}
\frametitle{GitHub link}

Organization used to manage our LaTeX presentation progression

\vspace{5mm}


\href{https://github.com/Research-Methods-Presentation}{GitHub organisation link} 

\end{frame}

%-----------------------------------------------

\begin{frame}
\Huge{\centerline{Questions?}}
\end{frame}

%-----------------------------------------------


\end{document} 